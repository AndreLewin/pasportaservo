\documentclass{article}

\usepackage{fontspec}
\usepackage{xunicode}
\usepackage{polyglossia}
\setdefaultlanguage{esperanto}
\setotherlanguage{korean}

% \usepackage{cmap}

\begin{document}

    \^h

Tio estas dokumento, kiu demonstru ke pere de Xe\LaTeX\ eblas skribi ĉiujn
literojn necesajn por Esperanto. Eĥoŝanĝo ĉiuĵaŭde; EĤOŜANĜO ĈIUĴAŬDE.

{{nomo}}

한국말

Preskaŭ ĉiuj latinalfabetaj lingvoj disponeblas: äöüß ÄÖÜ; àéê ąęįų ė żś,
ktp. Principe ĉiuj konataj alfabetoj eblas, se la tiparo havas la glifojn.

\section{ASCII English}
Hello world.
\section{European}
¡Hola!, Grüß Gott, Hyvää päivää, Tere õhtust, Bonġu
          Cześć!, Dobrý den, Здравствуйте!, Γειά σας, გამარჯობა
\section{CJK}
(Chinese) 你好, 早晨, (Japanese)こんにちは, (Korean, hangul) 안녕하세요

\end{document}
