\section{Kiel uzi la adreslibron?}
Serĉante gastiĝon oni sekvu la kondiĉojn de la gastiganto. Kelkaj gastigantoj pretas
akcepti gastojn sen antaŭa anonco, aliaj postulas ke oni antaŭe kontaktu ilin por
interkonsento. Kelkaj postulas skriban peton, aliaj akceptas (ankaŭ) telefonan
peton, aliaj retadresan.

En la adreslibro kutime aperas indikoj pri tio, kiom da gastoj povas akcepti la
gastiganto, kaj kiom longe la gasto(j) rajtas resti. Gastiganto principe devas akcepti
gastojn kiuj agis laŭ la koncernaj indikoj en la adreslibro, krom se tute ne konvenas.
Gastigado devas inkluzivi almenaŭ unu senpagan tranoktadon. Pri eventuala
manĝado oni faru interkonsenton. Se oni interkonsentas ke la gasto manĝos ĉe
la gastiganto, la gastiganto rajtas peti repagon de la kostoj faritaj por la manĝo
de la gasto. Gastiganto ne devas proponi manĝojn; gasto tiukaze mem aranĝu
siajn manĝojn. Gasto kompreneble ankaŭ povas proponi (prepari) manĝon por la
gastiganto. Kaj komuna kuirado estas agrabla!

Krom precizajn informojn pri la dezirata gastigado, ĝentilas enmeti internacian
respondkuponon (irk), aŭ du por aerpoŝta respondo. Internaciaj respondkuponoj
haveblas en poŝtoficejoj (kvankam bedaŭrinde ne plu en ĉiuj landoj). Se vi skribas
al pluraj personoj en la sama urbo, vi helpas al la gastigantoj se vi tion mencias. Kaj
se pro ŝanĝoj en feriaj planoj vi tamen ne povos veni al iu gastiganto, estas dece
almenaŭ tion sciigi laŭeble frue.

Por Pasporta Servo, oni ne plu akceptas internaciajn respondkuponojn.
\subsubsection{Ordigo kaj mapoj}
Landoj (jure kaj/aŭ fakte sendependaj ŝtatoj) aperas en alfabeta sinsekvo. Dekstre
sub la landotitolo vi trovos la mapon kiu montras la landon.
Ene de la landoj, adresoj estas listigitaj laŭ urbo. Dekstre de la urbonomo vi trovos
la geografiajn koordinatojn per kiuj ni produktas la mapojn. Malgrandaj vilaĝoj, kaj
urboj kiuj situas tre proksime al grandaj (internacie konataj) urboj estas ofte listigitaj
sub tiu granda urbo. Tio i.a. plifaciligas al vojaĝantoj trovi la gastigantojn.
\subsubsection{Loka Peranto}
Peras gastigojn en sia loko (urbo) kaj kontaktigas gastigantojn kun vojaĝantoj.
Li/ŝi eventuale eĉ konas adresojn de gastigantoj kiuj ne aperas en la adreslibro.
\subsubsection{Landa Organizanto}
Varbas gastigantojn en sia lando, i.a. pere de la oficiala varbilo, informas pri
Pasporta Servo (foje ankaŭ al ne-Esperantaj organizaĵoj), kaj helpas la kompilanton
diversmaniere. Li aŭ ŝi eble organizas rondvojaĝojn aŭ donas informojn (ekz. pri
sia lando) al individuaj vojaĝemuloj. Kelkaj Landaj Organizantoj ankaŭ vendas
ekzemplerojn de Pasporta Servo, aŭ povas informi pri (nacilingvaj) varbiloj pri
Esperanto aŭ Pasporta Servo. Adresojn de Landaj Organizantoj vi trovos komence
ĉe la respektivaj landoj en la adreslibro. Se iu lando ne havas Landan Organizanton
la kompilanto kaj la estraro de TEJO eventuale povas serĉi iun. Se laŭ vi iu lando
sen Landa Organizanto bezonas tian personon, skribu al Sylvie, la ĉeforganizantino
de Landaj Organizantoj (vidu sub Francio).
\subsubsection{https//:pasportaservo.org}
La oficiala retpaĝo de Pasporta Servo. Tie vi trovos aldonajn servojn, kiujn ne
haveblus kun adreslibro: zomebla monda mapo, serĉilo, forumo kaj blogoj. Uzantoj
povas rapide ŝanĝi siajn datenojn, sen atendi la aperon de la venonta adreslibro.
\subsubsection{Kompilanto}
Aktivulo de TEJO kiu kompilas la adreslibron de Pasporta Servo. Al li vi sendu
viajn aliĝilojn kaj memorigilojn. Li povas sendi al vi ankaŭ varbilojn (kun aliĝilo), kaj
ĉiajn informojn pri Pasporta Servo.
2011	
\subsubsection{Centra Distribuanto}
UEA, la ĉefa vendanto kaj distribuanto de la adreslibro. UEA havas krome ankaŭ
stoketon da varbiloj por Pasporta Servo (kun aliĝilo). Adreso aperas en la kolofono.
Kiel ricevi la adreslibron de gastigantoj?

Gastigantoj en la adreslibro rajtas aŭtomate ricevi senpagan ekzempleron kaj do
ne bezonas mendi ĝin. Preskaŭ ĉiu libroservo disponas pri stoko da adreslibroj,
sed ĉiam eblas mendi rekte de UEA (adreso en la kolofono). Krome vendas la
adreslibron kelkaj Landaj Organizantoj (vidu sub la koncerna lando), kaj kelkaj
Landaj Sekcioj de TEJO (pri ties adresoj informiĝu ĉe UEA, vidu supre).
\subsubsection{Mi ricevas tro da gastoj!}
Gastigantoj en grandaj urboj aŭ en turismaj lokoj kelkfoje ricevas pli da petoj ol
ili povas akcepti. Al tiuj gastigantoj ni konsilas ne eksiĝi, sed simple ne mencii
la straton (kaj tele­fonnumeron) en sia adreso. Anstataŭe menciu poŝtkeston,
faksnumeron aŭ retpoŝtan adreson, kun kelksemajna antaŭa sciigo. Tiumaniere oni
evitas neanoncitajn gastojn, kaj povas pli facile ne akcepti gaston. Tamen atentu
ke necesas funkcianta poŝtadreso por ricevi la adreslibron. Ankaŭ eblas aperi kiel
Loka Peranto, kiu distribuas petojn.

\subsubsection{Mi neniam ricevas gastojn!}
Uzu frapan rimarkon ĉe via propra adreso por klarigi al la gasto kial nepre necesas
viziti vin. Proponu allogan komenton por via lando, kiu klarigas kial indas viziti vian
landon. Kontaktu aliajn gastigantojn viaregione, por ekz-e kune oferti dusemajnan
itineron.
\subsubsection{Helpu nin!}
Ne ĉiuj informoj en Pasporta Servo ĝustas. Ekzistas (malgraŭ ĉio) neaktualaj
gastigantoj, kaj alispecaj misoj kiel tajperaroj, penseraroj kaj krokodiloj. Ni ne havas
eblon ĉion zorgege kontroli. Tial, se vi trovas ion malĝustan aŭ plibonigeblan,
ne hezitu! Skribu tuj al la kompilanto, kiu esploros la aferon kaj agos laŭcirkonstance.
\subsubsection{Telefono kaj poŝto}
\begin{description}
    \item[Telefonado] La menciitaj numeroj aperas en enlanda kaj internacia formoj. Regionaj
antaŭnumeroj, se ili ekzistas, aperas inter (krampoj). Ili ĝenerale estas kompletaj se
ili komenciĝas per ‘0’. Por telefoni el eksterlando, unue elektu la ellandan numeron
de via lando, poste la landnumeron de la alia lando, kaj poste la indikitan numeron
sen la komenca ‘0’. Se ne aperas regiona antaŭnumero, elektu la kompletan
numeron. Estas kelkaj esceptoj: pri tiuj vidu ekzemple la rubrikon ‘Telefonado’ en la
Jarlibro de UEA, kaj informiĝu pri la ĝustaj kodoj kiam vi estas surloke.

    \item[Poŝtkodoj] Ĝenerale oni metas la poŝtkodon antaŭ la urbonomo, kaj antaŭ la
poŝtkodo oni metas la landokodon. Principe ĉe tiuj landoj kiuj havas alian kutimon
la ĉi-koncernaj reguloj aperas tuj post la landonomo.
Landokodoj: Ni uzas la internacie rekonatan duliteran ISO-norman sistemon, kiun
ankaŭ UEA uzas en la Jarlibro.
\end{description}